\documentclass[13pt]{book}
% list of package used in 
\usepackage{amsmath}
\usepackage{amssymb}
\usepackage{mathtools}
\usepackage[ampersand]{easylist}
\usepackage{hyperref}

\title{Notes on Mathematics}
\author{Rangarajan R}
\date{\today}


\begin{document}
\maketitle

\tableofcontents

\chapter{Probability}
\section{Introduction}
\begin{itemize}
\item \textbf{Probability : } The prediction of a certain outcome when something occurs.
\item Examples :
  \begin{enumerate}
  \item Toss a die
  \item Flip a coin
  \item Draw a card
  \end{enumerate}
\end{itemize}
\subsection{Types of probabilities}
\begin{enumerate}
\item \textbf{Theoretical Probability : } Based on predictable parameters. \\
  \(\rightarrow\) P(Tossing a five) = \(\frac{1}{6}\) \\
  \(\rightarrow\) P(Drawing a spade) = \(\frac{1}{4}\) 
\item \textbf{Empirical Probability : } Based on historical and geological records. \\
  \(\rightarrow\) P(Strong earthquake in the next ten years) \\
  \(\rightarrow\) P(Getting into a car accident)
\item \textbf{Subjective Probability : } Based on experience or initution. \\
  \(\rightarrow\) P(Getting hurt when falling off a bicycle)
\end{enumerate}
\section{Definition}
\subsection{Sets and Elements}
\begin{itemize}
\item \textbf{Set} is composed of \textbf{elements or members}
\item where each element is a possible outcome \\
  \( A = \{a,b,c,d\}\) a \(\in\) A\\
  \( B = \{e,f,g,h\}\) a \(\ni\) B
\item A set can be defined in two ways
  \begin{enumerate}
  \item Listing all the elements \( A = \{a,b,c,d\}\)
  \item Describing the properties held by the members and/or non-members
    \end{enumerate}
\end{itemize}
\subsection{Sample spaces}
\begin{itemize}
\item \textbf{Sample Space (\(\Omega\)) : } A set whose elements describe the outcomes of the experiment of interest.
\item Examples : \\ Coin Toss \(\Omega\) = \( \{ head,tail \}\)\\
  Determine the month a student is born in \(\Omega\) = \( \{Jan,Feb,Mar\}\)
\end{itemize}
\subsection{Factorial}
\begin{itemize}
\item Determine the various ways (and the number of ways) 3 coloured marbles can be placed.
\item Example : \\
  \(\Omega\) = \(\{RBG,RGB,BGR,BRG,GBR,GRB\}\)
\item number of elements = n! = 3! = 1.2.3 = 6
\item And a \(y^{th}\) color (Y)
\item\(\Omega\) = \({YRBG,YRGB,...}\)
\item number of elements = n! = 4! = 1.2.3.4 = 24
\end{itemize}
\subsection{Events}
\begin{itemize}
  \item \textbf{Event} is a subset of a sample space
  \item Toss a die \(\Omega\) = \(\{1,2,3,4,5,6\}\)
  \item Event is tossing an even number = \({2,4,6}\)
  \item P(Event) = \(\frac{3}{6}\) = \(\frac{1}{2}\)
\end{itemize}
\subsection{Intersection, union, compliment}
\begin{itemize}
\item \(\Omega\) = \(\{1,2,3,4,5,6,7,8,9,10\}\) \\
  A = \(\{1,3,5,7,9\}\) \\
  B = \(\{2,4,6,8,10\}\) \\
  C = \(\{1,2,3,4,5\}\) \\
  D = \(\{4,8\}\) 
\item \textbf{Intersection : } A \(\cap\) C = \(\{1,3,5\}\)
\item \textbf{Union : } A \(\cup\) C = \(\{1,2,3,4,5,7,9\}\)
\item \textbf{Compliment : } \(A^C\) = \(\{2,4,6,8,10\}\)
\item \textbf{Disjoint :} A and D are disjoint
\item \textbf{Subset : } D \(\subset\) C
\item \textbf{Difference : }  B - D = \(\{2,6,10\}\)
\end{itemize}
\subsection{DeMorgan's Laws}
\begin{itemize}
  \item \(\Omega\) = \(\{1,2,3,4,5,6,7,8,9,10\}\) \\
  A = \(\{1,3,5,7,9\}\) \\
  B = \(\{2,4,6,8,10\}\) \\
  C = \(\{1,2,3,4,5\}\) \\
  D = \(\{4,8\}\)
    \item Laws :
  \begin{enumerate}
    \item \(A \cup C = A^C \cap C^C\)
    \item \(A \cap C = A^C \cup C^C\)
    \end{enumerate}
\end{itemize}
\subsection{Probability Function}
\begin{itemize}
\item Probability function, P, it assign number from 0 to 1 to each event in \(\Omega\) such that 
  \item P(\(\Omega\)) = 1
  \item P(A) = Number of ways \textbf{ A } can occur/total number of outcomes
  \item Disjoint, P(A\(\cup\)B) = P(A) + P(B)
  \item Overlapping, P(A\(\cup\)B) = P(A) + P(B) - P(A\(\cap\)B)
  \item Flipping \textbf{n} coins, \(\Omega\) = \(\{ 2^n\}\) outcomes
  \item Let k = number of heads (where n \(\preceq\) k \(\preceq\) 0)
  \item P(k heads, n-k tails) = \(\frac{\binom{n}{k}}{2^n}\) \\
    where \(\binom{n}{k} = \frac{n!}{k!(n-k)!}\)
  \item Ex n = 4, k = 1, (4-coins), P(1-H,3-T) = \(\frac{\frac{1.2.3.4}{(1)(1.2.3)}}{16}\) = \(\frac{1}{4}\)
  \item The probability of an event not occuring, P(Not A) = 1 - P(A)
  \item The probability of A or B (Independent Events) \\
    P(A or B) = P(A) + P(B)
  \item The probability of A or B (Dependent Events) \\
    P(A or B) = P(A) + P(B) - P(A\(\cap\)B)
  \end{itemize}
\subsection{}
\begin{itemize}
  \item
  \end{itemize}
\subsection{}
\begin{itemize}
  \item
\end{itemize}  
\section{Mathematics of Randomness}
\subsection{Probability}
\begin{itemize}
  \item \textbf{Equally possible outcomes} two or more events have equal chances of occuring
  \item The probability of an event is the ratio of the number of favourable outcomes to the total number of equally possible exclusive outcomes.
  \item Suppose \textbf{\(P_A\)} is the probability of an event \textbf {A}, \textbf{\(m_A\)} is the number of favourable outcomes, and \textbf{n} is the total number of equally possible and exclusive outcomes. 
    \[P_A = m_A/n\]
  \item If \(m_A=n\), then \(P_A = 1\) and the event A is a certain event (it always occurs in every outcome).
  \item If \(m_A=0\), then \(P_A= 0\) and the event A is an impossible event (it never occurs).
  \item The probability of random event lies between 0 and 1.
  \item Let an event \textbf{A} be throwing a die and getting exactly divisible by three(\(m_A\) = 2 and n = 6).
    \[P_A = 1/3\]
  \item A bag with 15 identical but differently coloured balls (seven white, two green and six red). \\
    The probability of drawing white ball, \( P_A = m_A/n = 7/15\) \\
    The probability of drawing green ball, \( P_B = m_B/n = 2/15\) \\
    The probability of drawing red ball, \( P_C = m_C/n = 2/5\)
  \item Rule for Addition of probabilities : the probability of one event of several exclusive events occuring is the sum of the probabilities of each separate event.

%    The probability that a randomly drawn ball will be either red or green, \(P_B_+_C = P_B + P_C\)
%    \[ = (m_B + m_C)/n = (6 + 2)/15 = 8/15\]
    
\end{itemize}
\subsection{Random Numbers}
\subsection{Random Events}
\subsection{Discrete Random Variables}
\subsection{Continuous Random Variables}
\section{Decision Making}
\chapter{Matrix Algebra}
\section{Basic Definition}
\begin{itemize}
\item A set \textbf{mn} of numbers (real or complex) arranged in a reactangular array of \textbf{m} rows and \textbf{n} columns
  \[A =
  \begin{pmatrix}
    a_{11} & a_{12} & a_{13} & \dots & a_{1n}\\
    a_{21} & a_{22} & a_{23} & \dots & a_{2n} \\
    \dots & \dots & \dots & \dots  & \dots \\
    a_{11} & a_{12} & a_{13} & \dots & a_{1n}
  \end{pmatrix}
\]
is called \textit{matrix} (of numbers).
\item The rows and columns above are termed the \textit{lines} of the matrix.
\item if m = n, the matrix is called a \textit{square matrix} of order \textit{n}.
\item if m \(\neq\) n, the matrix is called a \textit{square matrix} of order \textit{n}.
\item A \textit{1 X n} matrix is called a \textit{row vector}.
\item An \textit{m X 1} matrix is called a \textit{column vector}.
\end{itemize}
\newcommand{\sectionbreak}{\clearpage}
\section{Operations Involving Matrix}
\subsection{Equality of matrices}
\(\bullet\) Two matrices A = [a\(_{ij}\)] and B = [b\(_{ij}\)] are considered equal, A = B, if they have the same dimensions, and the corresponding elements are equal.
\subsection{The sum and difference of matrices}
\subsection{Multiplication of a matrix by a scalar}
\subsection{Multiplication of matrices}
\section{The transpose of a matrix}
\section{The inverse matrix}
\section{Power of a matrix}
\section{Rational function of a matrix}
\section{The absolute value and norm of a matrix}
\section{The rank of a matrix}
\section{The limit of a matrix}
\section{Series of matrix}
\section{Partitioned matrix}
\section{Matrix inversion by partitioned}
\section{Triangular matrix}
\section{Elementary transformations of matrices}
\section{Computation of determinants}
\part{Statistics}
\chapter{Basic Concepts}
\section{Population and Sample Data}
\subsection{Population Data}
\begin{itemize}
  \item Oftentimes we need to ask questions about and collect data for an entire group, class or type of people or objects.
  \item The population is defined by the researcher.
  \item There are no set rules to apply when defining a population except knowledge, common sense, and judgement.
  \item Population can be quite large.
\end{itemize}
\subsection{Sample Data}
\begin{itemize}
  \item When we need to make conclusions or estimates about an entire population we almost always use a sample(s) from our population of interest.
  \item A small but well-chosen sample can accurately reflect the characteristics of the entire population from which it is chosen.
  \item All elements in a sample must also by definition be part of the population as it is defined.
  \item The sample should be representative of the population from which it is drawn.
  \item In almost all cases(if not all cases) samples from the same population should be independent of each other.
  \item In many cases, the sample is chosen randomly from the population(random sample); but there are other methods.
  \item A sample is always an approximation of the population.
\end{itemize}

\begin{itemize}
  \item A
\end{itemize}

\begin{itemize}
  \item A
\end{itemize}

\end{document}
